{\bfseries  (C) Copyright 2010-\/2013 E. Brady Trexler and Kladiusz R. Weiss} ~\newline
~\newline
 ( \href{http://www.gothamsci.com}{\tt http\+://www.\+gothamsci.\+com} , ebtrexler (at) gothamsci.\+com )

( \href{http://www.mountsinai.org/profiles/klaudiusz-r-weiss}{\tt http\+://www.\+mountsinai.\+org/profiles/klaudiusz-\/r-\/weiss} )

~\newline
~\newline
 Coding practices\+:

This is a C++ application.

Classnames are prefixed with T, as in T\+Class\+Name, to denote a new Type. Most class data members have private access and are \char`\"{}got\char`\"{} and \char`\"{}set\char`\"{} by the \+\_\+\+\_\+property (closure) keyword, or public get and set methods. Private data members are prefixed with F for Field (eg. F\+Name).

Sotware Divisions\+:
\begin{DoxyEnumerate}
\item Class framework for cellular connections that handles wiring up network and saving/loading from disk. Future expansion of model with derived classes does not need to be concerned with plumbing, only the math and calculations.
\item Class framework(s) for data acquisition. These classes will provide public interfaces that are the same regardless of the hardware and drivers used.
\item Class framework for user interface. Currently implemented in Borland (Embarcadero) C++ Builder (R\+A\+D Studio 2010)
\end{DoxyEnumerate}

~\newline
~\newline
 \hypertarget{index_RTFramework}{}\section{$\ast$$\ast$$\ast$$\ast$$\ast$$\ast$$\ast$\+Overview of Real Time Network Class Framework$\ast$$\ast$$\ast$$\ast$$\ast$$\ast$$\ast$}\label{index_RTFramework}
Implements a set of serializable base classes that describe a cellular network. Each class has a member function called Update that takes a time step (in ms) (and if necessary the membrane voltages (in m\+V) of the cell or the pre and post synaptic cells), and returns the current to inject (in n\+A).

Design considerations\+:~\newline

\begin{DoxyItemize}
\item The class framework is designed to be independent of data acquisition hardware.
\item The class framework is designed to be thread safe.
\item Derived classes can implement calculations as desired.
\item The class framework is designed to be used with a G\+U\+I, but does not specify how it should be implemented.
\item Object creation and destruction (memory management) occurs by use of boost\+::shared\+\_\+ptr.
\item Applications that use this framework should keep with the convention of
\begin{DoxyItemize}
\item milli\+Volts (m\+V)
\item nano\+Amperes (n\+A)
\item micro\+Siemens (u\+S)
\item nano\+Farads (n\+F) and
\item milliseconds (ms)
\end{DoxyItemize}as units for computation.
\end{DoxyItemize}

{\bfseries  Class organization } 
\begin{DoxyPre}
       \hyperlink{class_t_network}{TNetwork}            // owns all others and the master Update function
       |      |            // handles object creation and deletion
     \hyperlink{class_t_cell}{TCell}    |
       |      |
  \hyperlink{class_t_synapse}{TSynapse} - \hyperlink{class_t_cell}{TCell}        // \hyperlink{class_t_cell}{TCell} owns \hyperlink{class_t_synapse}{TSynapse} and vice versa
       |      |  |
    \hyperlink{class_t_current}{TCurrent}  | \hyperlink{class_t_current}{TCurrent}   // derive from \hyperlink{class_t_current}{TCurrent} to implement new flavors
              |            // synapses and cells have \hyperlink{class_t_current}{TCurrent} arrays
              |
           \hyperlink{class_t_electrode}{TElectrode}      // Allows injection of arbitrary current
\end{DoxyPre}
\hypertarget{index_Derived}{}\subsection{\#\#\#\#\+Derived classes implementing network functionality\#\#\#\#}\label{index_Derived}
Currents\+:~\newline
 \hyperlink{class_t_h_h_kinetics_factor}{T\+H\+H\+Kinetics\+Factor} ~\newline
 \hyperlink{class_t_h_h_current}{T\+H\+H\+Current} ~\newline
 \hyperlink{class_t_h_h2_current}{T\+H\+H2\+Current} ~\newline
 \hyperlink{class_t_gap_junction_current}{T\+Gap\+Junction\+Current} ~\newline
 Calculations for above currents are based on dynamicclamp software from Farzan Nadim's laboratory~\newline
 ( \href{http://stg.rutgers.edu/farzan/}{\tt http\+://stg.\+rutgers.\+edu/farzan/} , farzan (at) njit.\+edu ~\newline
 -- See Tohidi and Nadim 2009 J Neurosci. and Willms et al 1999 J Comp Neurosci. ~\newline
~\newline
 \hyperlink{class_t_voltage_clamp___p_i_d___current}{T\+Voltage\+Clamp\+\_\+\+P\+I\+D\+\_\+\+Current}~\newline
 ~\newline
 ~\newline
 Synapses\+:~\newline
 \hyperlink{class_t_gen_bi_dir_synapse}{T\+Gen\+Bi\+Dir\+Synapse} ~\newline
 \hyperlink{class_t_gap_junction_synapse}{T\+Gap\+Junction\+Synapse} ~\newline
 ~\newline
 Cells\+:~\newline
 \hyperlink{class_t_biological_cell}{T\+Biological\+Cell} ~\newline
 \hyperlink{class_t_model_cell}{T\+Model\+Cell} ~\newline
 \hyperlink{class_t_playback_cell}{T\+Playback\+Cell}~\newline
 ~\newline
 Electrodes\+: ~\newline
 \hyperlink{class_t_injection_electrode}{T\+Injection\+Electrode} ~\newline
 \hyperlink{class_t_playback_electrode}{T\+Playback\+Electrode} ~\newline
 